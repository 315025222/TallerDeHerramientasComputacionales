\documentclass[letterpaper, 12pt, oneside]{article}%para dar formato al documento
\usepackage{amsmath}
\usepackage[spanish]{babel}
\usepackage{graphicx}
\usepackage{xcolor}
\usepackage[utf8]{inputenc}
\usepackage{enumitem}
\title{Problema 02}
\author{Carlos Emiliano Sandoval Amador \\ 315025222}
\date{24 de enero de 2019}
\begin{document}
	\maketitle
	\begin{center}
		\textbf{\large Suma de filas y columnas de una matriz.}
	\end{center}
	Utilice listas anidadas para representar la matriz, los elementos de la lista son las filas y los elementos de los elementos las columnas. Primero hice la suma de filas usando un doble ciclo for, el primero entra en las filas y el segundo itera sobre los elementos de esa fila, luego el resultado de la suma se guarda en una variable que se almacena en una lista que contiene la suma de cada fila. Para la suma de columnas hice lo mismo solo que empecé iterando sobre las columnas sabiendo que la longitud de cada fila es igual y luego iteré sobre las filas, Al final regreso una tupla con la lista de la suma de las filas y la lista de la suma de las columnas.
\end{document}