\documentclass[letterpaper, 12pt, oneside]{article}%para dar formato al documento
\usepackage{amsmath}
\usepackage[spanish]{babel}
\usepackage{graphicx}
\usepackage{xcolor}
\usepackage[utf8]{inputenc}
\usepackage{enumitem}
\title{Problema 01}
\author{Carlos Emiliano Sandoval Amador \\ 315025222}
\date{24 de enero de 2019}
\begin{document}
	\maketitle
	\begin{center}
		\textbf{\large Comparación de listas.}
	\end{center}
	Checa si dos listas son iguales, primero evaluando si tienen la misma cantidad de elementos con len() y luego, viendo si cada elemento de la primera lista es igual al que está en la misma posición en la segunda lista. Devuelve True si son iguales y False si no lo son.
\end{document}