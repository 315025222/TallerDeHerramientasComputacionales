\documentclass[letterpaper, 12pt, oneside]{article}%para dar formato al documento
\usepackage{amsmath}
\usepackage[spanish]{babel}
\usepackage{graphicx}
\usepackage{xcolor}
\usepackage[utf8]{inputenc}
\usepackage{enumitem}
\title{Problema 08}
\author{Carlos Emiliano Sandoval Amador \\ 315025222}
\date{24 de enero de 2019}
\begin{document}
	\maketitle
	\begin{center}
		\textbf{\large Loteria}
	\end{center}
	Mi programa importa la biblioteca random, de ahi utilizamos la función randrange, que escoge un número al azar de un rango del uno al 10, luego le pide un número al usuario, si los números son iguales, imprime "Has ganado un auto", sino "No has ganado nada". No fue posible hacerlo con listas, aunque de por sí el rango de donde toma un número aleatorio ya es una lista.
\end{document}