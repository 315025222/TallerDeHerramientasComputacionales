\documentclass[letterpaper, 12pt, oneside]{article}%para dar formato al documento
\usepackage{amsmath}
\usepackage[spanish]{babel}
\usepackage{graphicx}
\usepackage{xcolor}
\usepackage[utf8]{inputenc}
\usepackage{enumitem}
\title{Problema 01}
\author{Carlos Emiliano Sandoval Amador \\ 315025222}
\date{24 de enero de 2019}
\begin{document}
	\maketitle
	\begin{center}
		\textbf{\large Máximo común divisor.}
	\end{center}
	Para el problema 1 utilicé listas desde el principio, hice una función que recorra todos los números naturales menores o iguales a un número y utilicé el módulo para saber si esos naturales dividian al número más grande y si sí lo dividian los guardaba en una lista. Después hice la función del mcd, la cual recibe dos números, saca las listas de sus divisores  con un for compara si estos divisores son iguales, si lo son, los guarda en una lista. El último elemento de esta lista es el que devuelve la función.
\end{document}