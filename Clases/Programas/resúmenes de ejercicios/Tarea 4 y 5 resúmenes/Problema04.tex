\documentclass[letterpaper, 12pt, oneside]{article}%para dar formato al documento
\usepackage{amsmath}
\usepackage[spanish]{babel}
\usepackage{graphicx}
\usepackage{xcolor}
\usepackage[utf8]{inputenc}
\usepackage{enumitem}
\title{Problema 04}
\author{Carlos Emiliano Sandoval Amador \\ 315025222}
\date{24 de enero de 2019}
\begin{document}
	\maketitle
	\begin{center}
		\textbf{\large Fibonacci}
	\end{center}
	Para la sucesión de Fibonacci utilicé un ciclo while que haga n veces la suma de fibonacci para poder encontrar el n-ésimo número de Fibonacci. Hacerlo con listas fue más útil y simple porque solo tengo que tomar los últimos dos elementos de una lista, para encontrar el siguiente número defibonacci además escribí menos líneas de código.
\end{document}