\documentclass[letterpaper, 12pt, oneside]{article}%para dar formato al documento
\usepackage{amsmath}
\usepackage[spanish]{babel}
\usepackage{graphicx}
\usepackage{xcolor}
\usepackage[utf8]{inputenc}
\usepackage{enumitem}
\title{Problema 09}
\author{Carlos Emiliano Sandoval Amador \\ 315025222}
\date{24 de enero de 2019}
\begin{document}
	\maketitle
	\begin{center}
		\textbf{\large Tablas de multiplicar.}
	\end{center}
	Imprime en consola las tablas de multiplicar, utilizo un ciclo while para recorrer los números del uno al diez y luego multiplico el índice por el número del cual quiero su tabla de multiplicar, después gurado los resultados en una variable y uso format para imprimir un string con toda la información. No era útil implementar listas.
\end{document}