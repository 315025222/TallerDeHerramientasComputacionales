\documentclass[letterpaper, 12pt, oneside]{article}%para dar formato al documento
\usepackage{amsmath}
\usepackage[spanish]{babel}
\usepackage{graphicx}
\usepackage{xcolor}
\usepackage[utf8]{inputenc}
\usepackage{enumitem}
\title{Problema 03}
\author{Carlos Emiliano Sandoval Amador \\ 315025222}
\date{24 de enero de 2019}
\begin{document}
	\maketitle
	\begin{center}
		\textbf{\large Números primos.}
	\end{center}
	Utilicé una función que definí en la tarea cuatro que encuentra los divisores de un número dado, por lo que un número es primo si la lista de sus divisores es de longitud dos, es decir, contiene al uno y al él mismo.
\end{document}