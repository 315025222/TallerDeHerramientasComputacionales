\documentclass[letterpaper, 12pt, oneside]{article}%para dar formato al documento
\usepackage{amsmath}
\usepackage[spanish]{babel}
\usepackage{graphicx}
\usepackage{xcolor}
\usepackage[utf8]{inputenc}
\usepackage{enumitem}
\title{Bitacora 04}
\author{Carlos Emiliano Sandoval Amador \\ 315025222}
\date{10 de enero de 2019}
\begin{document}
	\maketitle
	Empezamos a utilizar el lenguaje de programación Python, el cual está hecho para resolver problemas de índole científica. Para poderlo utilizar de mejor manera requerimos de una ide llamada idle.\\ \\ Comenzamos aprendiendo a imprimir datos en la consola utilizando la función \textbf{print()}. Para utilizarla solo tenemos que ingresar adentro de los paréntesis el tipo de dato que queramos imprimir, es decir, string, entero, flotante, lista o tupla. \\ \\ Después modelamos y resolvimos un problema propuesto por el profesor de mecánica clásica. El problema nos daba la velocidad y aceleración del objeto y nos pedía encontrar la posición del objeto en un determinado tiempo. Para ello utilizamos la siguiente ecuación de la teoría de la física clásica.
	\begin{equation} %hice la ecuación de distancia recorrida en tiro vertical y caída libre
		y = v_0t - \frac{1}{2}gt^2
	\end{equation}
	Para asignar un valor a una variable debemos escribir el nombre de la variable seguido de \textbf{=} y el valor que deseamos asignar. Si queremos sumar números tenemos que usar \textbf{+}, para restar \textbf{-}, para multiplicar \textbf{*}, para dividir \textbf{/} y para un exponente \textbf{**}. \\ \\ Para resolver el problema asignamos la velocidad a una variable llamada v\textsubscript{0}, el tiempo a una variable llamada t y la aceleración a una variable llamada g. Por último escribimos la ecuación de arriba en idle y le asignamos su valor a una variable y, la cual imprimimos en consola.

\end{document}