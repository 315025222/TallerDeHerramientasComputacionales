\documentclass[letterpaper, 12pt, oneside]{article}%para dar formato al documento
\usepackage{amsmath}
\usepackage[spanish]{babel}
\usepackage{graphicx}
\usepackage{xcolor}
\usepackage[utf8]{inputenc}
\usepackage{enumitem}
\title{Bitácora 09}
\author{Carlos Emiliano Sandoval Amador \\ 315025222}
\date{17 de enero de 2019}
\begin{document}
	\maketitle
	\textbf{\Large Latex.} \\ En la clase de hoy aprendimos como hacer un libro con índice, capítulos y bibiliografía, además utilizamos el paquete babel con la opción español para que los contenidos del libro que se generan automáticamente aparezcan en español. \\ Descubrimos una nueva biblioteca de python llamada os, con cual podemos ejecutar comandos de bash desde nuestro programa de python. También conocimos nuevas funciones de la biblioteca math como \textbf{sinh()} y valores aproximados de e y $\pi$, además sinh está definida también como $sinh(x) = \frac{1}{2}(e^{x}-e^{-x})$, sin embargo existen bibliotecas de python más precisas pues con los valores de math, los resultados de la función sinh y su definición en términos de e no son iguales a partir de un número determinado de decimales.

\end{document}