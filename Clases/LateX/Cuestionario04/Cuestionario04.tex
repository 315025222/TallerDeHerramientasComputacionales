\documentclass[letterpaper, 12pt, oneside]{article}%para dar formato al documento
\usepackage{amsmath}
\usepackage[spanish]{babel}
\usepackage{graphicx}
\usepackage{xcolor}
\usepackage[utf8]{inputenc}
\usepackage{enumitem}
\title{Cuestionario 04}
\author{Carlos Emiliano Sandoval Amador \\ 315025222}
\date{22 de enero de 2019}
\begin{document}
	\maketitle
	\begin{enumerate}
		\item ¿Cómo podemos hacer sublistas en python a partir de una lista? \\ Podemos hacerlo de tres formas distintas, de un índice a otro índice, desde que empieza la lista hasta un determinado índice y desde un índice hasta que termina la lista.
		\item ¿Cómo hacemos una sublista de un índice a otro? \\ Para hacer una sublista de un índice a otro tenemos que colocar el nombre de la lista seguido de [ ] y dentro ponemos el índice de inicio seguido de dos puntos y el índice final, por ejemplo, lista[2:5].
		\item ¿Cómo hacemos una sublista desde que empieza la lista hasta un determinado índice? \\ Para hacer una sublista desde que empieza hasta un determinado índice tenemos que colocar el nombre de la lista seguido de [ ] y dentro ponemos dos puntos seguido del índice donde queremos que termine, por ejemplo, lista[:6].
		\item ¿Cómo hacer una sublista desde un índice hasta que acaba la lista? \\ Para hacer una sublista desde un índice hasta que acaba la lista tenemos que colocar el nombre de la lista seguido de [ ] y dentro ponemos el índice de inicio seguido de dos puntos, por ejemplo, lista[3:].
		\item ¿Cuál es la diferencia entre == y is? \\ Con == comparamamos si dos variables contienen los mismos valores pero con is, las variables deben contener los mismos valores y estar guardadas en el mismo espacio en la memoria.
	\end{enumerate}
\end{document}