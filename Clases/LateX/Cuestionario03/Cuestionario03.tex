\documentclass[letterpaper, 12pt, oneside]{article}%para dar formato al documento
\usepackage{amsmath}
\usepackage[spanish]{babel}
\usepackage{graphicx}
\usepackage{xcolor}
\usepackage[utf8]{inputenc}
\usepackage{enumitem}
\title{Cuestionario 03}
\author{Carlos Emiliano Sandoval Amador \\ 315025222}
\date{25 de enero de 2019}
\begin{document}
	\maketitle
	\begin{enumerate}
		\item ¿De qué tipo son los datos que devuelve input()? \\ De tipo string.
		\item ¿Para que sirve enumerate? \\  \textbf{enumerate} nos regresa una lista cuyos valores son tuplas de longitud dos cuya primer entrada es el índice y la segunda entrada el valor de la lista en ese índice.
		\item ¿Para que sirve zip? \\ 	 \textbf{zip())} nos permite iterar sobre más de un objeto iterable a la vez. Para usarla tenemos que colocar un for y el mismo número de índices y de objetos iterables, los objetos iterables van dentro de \textbf{zip()} separados por comas.
		\item ¿Cómo se declaran listas usando for pero sin usar append? \\ Tenemos que poner el ciclo for entre corchetes, una expresión matemática que involucre al índice i que será el valor de la lista en la posición i-1, el índice i y el objeto iterable. 
		\item ¿Para que sirven los paréntesis? \\ Los paréntesis sirven para declarar tuplas de manera explícita en python.
		\item ¿Para que sirven las llaves? \\ Las llaves sirven para declarar diccionarios o conjuntos en python.
		\item ¿Para que sirven los corchetes? \\ Los corchetes sirven para declarar listas en python de manera explícita o para acceder al valor de una lista o una tupla a partir de un índice.
	\end{enumerate}
\end{document}