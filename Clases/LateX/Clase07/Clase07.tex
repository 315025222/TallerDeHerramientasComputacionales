\documentclass[letterpaper, 12pt, oneside]{article}%para dar formato al documento
\usepackage{amsmath}
\usepackage[spanish]{babel}
\usepackage{graphicx}
\usepackage{xcolor}
\usepackage[utf8]{inputenc}
\usepackage{enumitem}
\title{Bitácora 07}
\author{Carlos Emiliano Sandoval Amador \\ 315025222}
\date{15 de enero de 2019}
\begin{document}
	\maketitle
	\begin{center}
		\textbf{\Large Estructuras de repetición.}
	\end{center}
	\textbf{While.} \\ Esta estructura de repetición funciona con dos cosas; una condición y un bloque de código. Primero evalua si la condición es verdadera, si es así, ejecuta el bloque, de lo contrario avanza a la siguiente línea de código e ignora el bloque. Cuando la condición es cierta y el bloque ha sido ejecutado entonces se repite el proceso anterior.  \\ \\ Los espacios sirven para identar bloques que ejecutan las estructuras de control y de repetición bajo determinadas condiciones.
\end{document}