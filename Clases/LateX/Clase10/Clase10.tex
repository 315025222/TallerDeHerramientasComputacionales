\documentclass[letterpaper, 12pt, oneside]{article}%para dar formato al documento
\usepackage{amsmath}
\usepackage[spanish]{babel}
\usepackage{graphicx}
\usepackage{xcolor}
\usepackage[utf8]{inputenc}
\usepackage{enumitem}
\title{Bitácora 10}
\author{Carlos Emiliano Sandoval Amador \\ 315025222}
\date{18 de enero de 2019}
\begin{document}
	\maketitle
	\section{\large Usando bool.}
	Con \textbf{bool()} podemos obtener True o False dependiendo del tipo de objeto que le demos. Con un string o una lista devuelve False si está vacía y True si tiene elementos. Con números devuelve True si no son cero y False en caso contrario.
	\section{\large Listas.}
	Para declarar una lista tenemos que utilizar [] y dentro ponemos los elementos de la lista separados por comas. Los elementos pueden ser cualquier otro objeto de python como strings, números, booleanos e incluso ¡otras listas!. \\ \\
	\section{Métodos de listas.}
	Las listas son una clase de python por ende tienen diversos métodos con los que podemos modificar, agregar o eliminar sus atributos. A continuación explicamos los métodos más importantes.
	\subsection{append()}
	Con \textbf{append()} podemos agregar un  nuevo elemento a una lista en el último lugar de esta. Para usarla solo tenemos que colo car el elemento deseado en el paréntesis.
	\subsection{insert()}
	Agrega un nuevo elemento a la lista en la posición que desees.Pide dos parámetros, la posición y el elemento.
	\subsection{remove()}
	Elimina el elemento que le indiquemos de la lista. El único parámetro que pide es el elemento que deseamos eliminar.
	\subsection{pop()}
	Con \textbf{pop()} podemos sacar un elemento de una lista y almacenarlo en una variable. Si no colocamos parámetro entonces se saca el último elemento de la lista. Podemos colocar la posición del elemento que queremos sacar de la lista.
	\subsection{len()}
	La función \textbf{len()} es de python y sirve para contar la longitud de una lista, es decir, cuántos elementos tiene.
	\section{Ciclo for}
	La palabra reservada \textbf{for} induce un ciclo de repetición con el cual podemos recorrer un arreglo en su totalidad. Para utilizar \textbf{for} tenemos que poner \textbf{for} seguido de una variable luego la palabra reservada \textbf{in} y un objeto iterable, es decir, una lista, un string, un rango, entre otros.
	
	
	
\end{document}