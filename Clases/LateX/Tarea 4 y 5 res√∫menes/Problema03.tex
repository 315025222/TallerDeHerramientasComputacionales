\documentclass[letterpaper, 12pt, oneside]{article}%para dar formato al documento
\usepackage{amsmath}
\usepackage[spanish]{babel}
\usepackage{graphicx}
\usepackage{xcolor}
\usepackage[utf8]{inputenc}
\usepackage{enumitem}
\title{Problema 03}
\author{Carlos Emiliano Sandoval Amador \\ 315025222}
\date{24 de enero de 2019}
\begin{document}
	\maketitle
	\begin{center}
		\textbf{\large Conversiones de grados.}
	\end{center}
	Hice dos funciones, una que convierte grados fahrenheit a grados celsius y otra que convierte grados celsius a grados fahrenheit, utilizando las ecuaciones de conversión y asignando el valor a una variable. Para hacerlo con listas solo pedí que devolvieran una lista con el resultado en lugar de un flotante.
	
\end{document}