\documentclass[letterpaper, 12pt, oneside]{article}%para dar formato al documento
\usepackage{amsmath}
\usepackage[spanish]{babel}
\usepackage{graphicx}
\usepackage{xcolor}
\usepackage[utf8]{inputenc}
\usepackage{enumitem}
\title{Problema 05}
\author{Carlos Emiliano Sandoval Amador \\ 315025222}
\date{24 de enero de 2019}
\begin{document}
	\maketitle
	\begin{center}
		\textbf{\large Suma}
	\end{center}
	Para hacer la suma utilicé un ciclo while e incremente una variable cada vez que entraba al ciclo, los valores de la suma los almacené en otra variable, esta variable es la que regresa la función al terminar el ciclo while. En el caso con listas aproveché para guradar cada término en una lista yu poder hacer un string que represente la suma y su resultado.
\end{document}