\documentclass[letterpaper, 12pt, oneside]{article}%para dar formato al documento
\usepackage{amsmath}
\usepackage[spanish]{babel}
\usepackage{graphicx}
\usepackage{xcolor}
\usepackage[utf8]{inputenc}
\usepackage{enumitem}
\title{Bitacora 01}
\author{Carlos Emiliano Sandoval Amador \\ 315025222}
\date{07 de enero de 2019}


\begin{document}
	\maketitle % hace el título
	{Lo que hice esta primera clase en el curso fue:}
	\begin{enumerate} % enumera los temas
		\item Conocer los siguientes sistemas operativos:
		\begin{enumerate} % enumera los sistemas
			\item ¡OS
			\item Linux
			\item Windows
		\end{enumerate} 
		\item Checar distintas distribuciones de Linux,
		estas son:
		\begin{enumerate} %enumera distribuciones
			\item Ubuntu
			\item Fedora
		\end{enumerate}
		\item Conocer lenguajes de programación como:
		\begin{enumerate} %enumera lenguajes
			\item Java
			\item Python
			\item C
			\item C++
			
		\end{enumerate}
		
	\end{enumerate} 
% prqueño resumen
{Después acordamos que el sistema operativo que ibamos a utilizar sería Linux pues es "Open source", es decir, puede ser modificado por cualquier persona siempre y cuando no sea con fines de lucro, por lo cual es un sistema operativo muy seguro y eficaz.\\ \\}
{Luego decidimos que el lenguaje de programación con el que trabajariamos sería Python porque es un lenguaje enfocado a la modelación y resolución de problemas científicos y es muy versátil sin dejar de lado la formalidad y la precisión. \\ \\}
{Por último el profesor nos explicó que navegariamos en la consola en lugar de utilizar la interfaz gráfica y que utilizariamos un shell, es decir, un interprete que nos permite utilizar los servicios del sistema operativo. En este caso ese shell es un intérprete de comandos llamado bash.}
	
\end{document}