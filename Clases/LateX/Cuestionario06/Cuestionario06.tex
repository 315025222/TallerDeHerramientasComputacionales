\documentclass[letterpaper, 12pt, oneside]{article}%para dar formato al documento
\usepackage{amsmath}
\usepackage[spanish]{babel}
\usepackage{graphicx}
\usepackage{xcolor}
\usepackage[utf8]{inputenc}
\usepackage{enumitem}
\title{Cuestionario 06}
\author{Carlos Emiliano Sandoval Amador \\ 315025222}
\date{25 de enero de 2019}
\begin{document}
	\maketitle
	\begin{enumerate}
		\item ¿Qué pasa si le aplicamos list a un string? \\ El resultado es una lista que contiene cada caracter de la cadena como un elemento de la lista, respetando el orden en el que aparecian en la cadena.
		\item ¿Cómo sabemos cuanto caracteres tiene un string? \\ Usando \textbf{len()}
		\item ¿Cómo obtenemos cada caracter de un string? \\ Utilizando un ciclo \textbf{for} donde el índice del ciclo toma el valor del i-ésimo caracter.
	\end{enumerate}
\end{document}