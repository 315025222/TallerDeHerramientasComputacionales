\documentclass[letterpaper, 12pt, oneside]{article}%para dar formato al documento
\usepackage{amsmath}
\usepackage[spanish]{babel}
\usepackage{graphicx}
\usepackage{xcolor}
\usepackage[utf8]{inputenc}
\usepackage{enumitem}
\title{Bitácora 06}
\author{Carlos Emiliano Sandoval Amador \\ 315025222}
\date{14 de enero de 2019}
\begin{document}
	\maketitle
	%Introducción.
	En la clase de hoy abordamos el problema de encontrar la raíz cuadrada de un número real no negativo. Aprendimos que no siempre podemos encontrar una solución exacta a un problema pero podemos encontrar una aproximación que satisfaga nuestras necesidades. \\
	\\ %título y ponemos en negritas palabras importantes
	 \textbf{Estructuras de control.} \\ Definimos la función valor absoluto utilizando las estructuras de control \textbf{if} y \textbf{else}. Para utilizar \textbf{if} tenemos que colocar una sentencia booleana y \textbf{:}, después presionamos enter y idle identa automáticamente el renglón de abajo, en este espacio se coloca lo que debe de hacer la computadora si la sentencia booleana es verdadera. Si la sentencia es falsa podemos colocar un \textbf{else} en el mismo nivel de identación y debajo de \textbf{if}, después colocamos \textbf{:} y presionamos enter, idle identa y en este bloque colocamos las instrucciones que sigue la computadora a continuación. \\ \\ 
	 
	 Creamos una función que simula un juego de diana utilizando estructuras de control.
\end{document}