\documentclass[letterpaper, 12pt, oneside]{article}%para dar formato al documento
\usepackage{amsmath}
\usepackage[spanish]{babel}
\usepackage{graphicx}
\usepackage{xcolor}
\usepackage[utf8]{inputenc}
\usepackage{enumitem}
\title{Bitácora 11}
\author{Carlos Emiliano Sandoval Amador \\ 315025222}
\date{21 de enero de 2019}
\begin{document}
	\maketitle
	\section*{input().}
	La clase de hoy vimos que al utilizar \textbf{input()} el valor que ingresa el usuario es de tipo string, por lo que tenemos que cambiar el tipo utilizando las funciones correspondientes. 
	\section*{enumerate().}
	También vimos la función \textbf{enumerate}, la cual nos regresa una lista cuyos valores son tuplas de longitud dos cuya primer entrada es el índice y la segunda entrada el valor de la lista en ese índice.
	\section*{zip().}
	Aprendimos a usar \textbf{zip())} la cual nos permite iterar sobre más de un objeto iterable a la vez. Para usarla tenemos que colocar un for y el mismo número de índices y de objetos iterables, los objetos iterables van dentro de \textbf{zip()} separados por comas.
	\section*{Listas con for.}
	Una de las grandes ventajas de python es la de poder definir una lista por medio de un ciclo for. Para definirla tenemos que poner el ciclo for entre corchetes, una expresión matemática que involucre al índice i que será el valor de la lista en la posición i-1, el índice i y el objeto iterable.
	\section*{Paréntesis.}
	Los paréntesis sirven para declarar tuplas de manera explícita en python.
	\section*{Corchetes.}
	Los corchetes sirven para declarar listas en python de manera explícita o para acceder al valor de una lista o una tupla a partir de un índice.
	\section*{Llaves.}
	Las llaves sirven para declarar diccionarios o conjuntos en python.
\end{document}