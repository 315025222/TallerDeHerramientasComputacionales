\documentclass[letterpaper, 12pt, oneside]{article}%para dar formato al documento
\usepackage{amsmath}
\usepackage[spanish]{babel}
\usepackage{graphicx}
\usepackage{xcolor}
\usepackage[utf8]{inputenc}
\usepackage{enumitem}
\title{Bitácora 14}
\author{Carlos Emiliano Sandoval Amador \\ 315025222}
\date{24 de enero de 2019}
\begin{document}
	\maketitle
	\begin{center}
		\textbf{\large El laberinto del python.} 
	\end{center}
	El día de hoy abordamos con detenimiento el problema de diseñar y programar un algoritmo recursivo que encuentre la salida de un laberinto. \\
	\begin{center}
		\textbf{\large Cadenas y listas.} 
	\end{center}
	Descubrimos que si utilizamos la función \textbf{list()} en un string, el resultado es una lista que contiene cada caracter de la cadena como un elemento de la lista, respetando el orden en el que aparecian en la cadena. Tambien podemos obtener la longitud de un string usando \textbf{len()} y acceder a cada caracte utilizando un ciclo \textbf{for} donde el índice del ciclo toma el valor del i-ésimo caracter.
\end{document}