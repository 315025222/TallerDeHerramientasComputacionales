\documentclass[letterpaper, 12pt, oneside]{article}%para dar formato al documento
\usepackage{amsmath}
\usepackage[spanish]{babel}
\usepackage{graphicx}
\usepackage{xcolor}
\usepackage[utf8]{inputenc}
\usepackage{enumitem}
\title{Cuestionario 02}
\author{Carlos Emiliano Sandoval Amador \\ 315025222}
\date{15 de enero de 2019}
\begin{document}
	\maketitle
	\begin{enumerate}
		\item ¿Cómo funciona if? \\ Para utilizar \textbf{if} tenemos que colocar una sentencia booleana y \textbf{:}, después presionamos enter y idle identa automáticamente el renglón de abajo, en este espacio se coloca lo que debe de hacer la computadora si la sentencia booleana es verdadera.
		\item ¿Cómo funciona else? \\ Si la sentencia es falsa podemos colocar un \textbf{else} en el mismo nivel de identación y debajo de \textbf{if}, después colocamos \textbf{:} y presionamos enter, idle identa y en este bloque colocamos las instrucciones que sigue la computadora a continuación.
		\item ¿Cómo funciona while? \\ ta estructura de repetición funciona con dos cosas; una condición y un bloque de código. Primero evalua si la condición es verdadera, si es así, ejecuta el bloque, de lo contrario avanza a la siguiente línea de código e ignora el bloque. Cuando la condición es cierta y el bloque ha sido ejecutado entonces se repite el proceso anterior.
		\item ¿Qué es importante saber cuando usamos while? \\ Es importante tener la certeza de que en un número finito de veces la condición será falsa de lo contrario el programa no funcionará de manera adecuada.
		\item ¿Para que sirven los espacios? \\ Los espacios sirven para identar bloques que ejecutan las estructuras de control y de repetición bajo determinadas condiciones. 
		\item ¿Cómo puedo colocar fórmulas matemáticas en latex? \\ Para colocar fórmulas matemáticas tenemos que encerrar la fórmula entre signos de peso o podemos utilizar begin con el entorno equation. Para más información domeos consultar Matemáticas en la barra de herramientas.
		\item ¿Cómo utilizo format? \\ Hay que colocar \textbf{.format} después del string seguido de un paréntesis dentro del cuál colocaremos el o los datos que queremos agregar, en lugar de usar porcentaje usamos llaves.
		\item ¿Qué ventaja tiene format sobre \%? \\ La ventaja que tiene \textbf{.format} sobre el porcentaje es que podemos agregar datos en orden diferente, pues podemos utilizar etiquetas para cada dato y podemos colocar un dato en más de una ocasión en el string con tan solo ponerlo una vez en \textbf{format}.
		\item ¿Qué es un objeto? \\ Un objeto es un conjunto de datos y funciones relacionadas que tiene atributos y métodos.
		\item ¿Qué es un método? \\ Un \textbf{método} es una acción que realiza el objeto.
		\item ¿Qué es un atributo? \\ Un \textbf{atributo} es una característica del objeto.
		\item ¿Qué es una clase? \\ Una clase es una plantilla para la creación de objetos de datos según un modelo predefinido. Cada clase es un modelo que define un conjunto de variables, los atributos y métodos apropiados para operar con dichos datos.
		\item ¿Cómo se llama el objeto de una clase? \\ Un objeto creado a partir de la clase se denomina instancia de la clase.
		\item ¿Cómo sabemos el tipo de un dato? \\ Para saber el tipo de dato de una variable podemos utilizar la función \textbf{type()}.
		\item  ¿Cómo cambiamos datos a entero? \\ Para cambiar datos a entero utilizamos \textbf{int()}.
		\item ¿Cómo cambiamos datos a flotante? \\ Para cambiar a flotante utilizamos \textbf{float()}.
		\item ¿Cómo cambiamos datos a string? \\ Para cambiar a string utilizamos \textbf{str()}.
		
	\end{enumerate}
\end{document}