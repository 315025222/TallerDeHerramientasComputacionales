\documentclass[letterpaper, 12pt, oneside]{article}%para dar formato al documento
\usepackage{amsmath}
\usepackage[spanish]{babel}
\usepackage{graphicx}
\usepackage{xcolor}
\usepackage[utf8]{inputenc}
\usepackage{enumitem}
\title{Cuestionario 02}
\author{Carlos Emiliano Sandoval Amador \\ 315025222}
\date{15 de enero de 2019}
\begin{document}
	\maketitle
	\begin{enumerate}
		\item ¿Cómo funciona if? \\ Para utilizar \textbf{if} tenemos que colocar una sentencia booleana y \textbf{:}, después presionamos enter y idle identa automáticamente el renglón de abajo, en este espacio se coloca lo que debe de hacer la computadora si la sentencia booleana es verdadera.
		\item ¿Cómo funciona else? \\ Si la sentencia es falsa podemos colocar un \textbf{else} en el mismo nivel de identación y debajo de \textbf{if}, después colocamos \textbf{:} y presionamos enter, idle identa y en este bloque colocamos las instrucciones que sigue la computadora a continuación.
		\item ¿Cómo funciona while? \\ ta estructura de repetición funciona con dos cosas; una condición y un bloque de código. Primero evalua si la condición es verdadera, si es así, ejecuta el bloque, de lo contrario avanza a la siguiente línea de código e ignora el bloque. Cuando la condición es cierta y el bloque ha sido ejecutado entonces se repite el proceso anterior.
		\item ¿Qué es importante saber cuando usamos while? \\ Es importante tener la certeza de que en un número finito de veces la condición será falsa de lo contrario el programa no funcionará de manera adecuada.
		\item ¿Para que sirven los espacios? \\ Los espacios sirven para identar bloques que ejecutan las estructuras de control y de repetición bajo determinadas condiciones.
		
	\end{enumerate}
\end{document}