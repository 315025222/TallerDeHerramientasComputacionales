\documentclass[letterpaper, 12pt, oneside]{article}%para dar formato al documento
\usepackage{amsmath}
\usepackage[spanish]{babel}
\usepackage{graphicx}
\usepackage{xcolor}
\usepackage[utf8]{inputenc}
\usepackage{enumitem}
\title{Bitácora 12}
\author{Carlos Emiliano Sandoval Amador \\ 315025222}
\date{22 de enero de 2019}
\begin{document}
	\maketitle %hace el título
	\section{Sublistas.} %dividimos el texto en dos secciones enumeradas y la primera sección tiene subecciones enumeradas
	Hoy aprendimos a hacer una sublista a poartir de una lista utilizando los índices. Para recorrer una lista podemos hacerlo de tres formas distintas, de un índice a otro índice, desde que empieza la lista hasta un determinado índice y desde un índice hasta que termina la lista.
	\subsection{De un índice a otro.}
	Para hacer una sublista de un índice a otro tenemos que colocar el nombre de la lista seguido de [ ] y dentro ponemos el índice de inicio seguido de dos puntos y el índice final, por ejemplo, lista[2:5].
	\subsection{Desde que empieza hasta un determinado índice.}
	Para hacer una sublista desde que empieza hasta un determinado índice tenemos que colocar el nombre de la lista seguido de [ ] y dentro ponemos dos puntos seguido del índice donde queremos que termine, por ejemplo, lista[:6].
	\subsection{Desde un índice hasta que acaba.}
	Para hacer una sublista desde un índice hasta que acaba la lista tenemos que colocar el nombre de la lista seguido de [ ] y dentro ponemos el índice de inicio seguido de dos puntos, por ejemplo, lista[3:].
	\section{Comparaciones.}
	% usamos \textbf para señalar palabras importantes.
	A veces necesitamos preguntarnos si dos variables contienen lo mismo o si son el mismo objeto, para saber si dos variables contienen lo mismo usamos \textbf{==} y para saber si son el mismo objeto utilizamos la palabra reservada \textbf{is}. Si \textbf{is} devuelve un valor verdadero significa que las variables son la referencia del mismo objeto, es decir, que están guardadas en el mismo lugar en la memoria.
\end{document}