\documentclass[letterpaper, 12pt, oneside]{article}%para dar formato al documento
\usepackage{amsmath}
\usepackage[spanish]{babel}
\usepackage{graphicx}
\usepackage{xcolor}
\usepackage[utf8]{inputenc}
\usepackage{enumitem}
\title{Bitácora 13}
\author{Carlos Emiliano Sandoval Amador \\ 315025222}
\date{23 de enero de 2019}
\begin{document}
	\maketitle
	\section*{Recursividad}
	En la clase de hoy aprendimos a definir una función de manera recursiva. Necesitamos dividir en dos casos ajenos, en el primero se da el valor que la función toma en el primer elemento y en el segundo caso el valor que toma la función en el n-ésimo valor en términos del n-1-ésimo valor. Si una función recursiva está bien definida entonces con un número finito de pasos siempre llegamos al paso base.
\end{document}