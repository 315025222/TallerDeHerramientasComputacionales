\documentclass[letterpaper, 12pt, oneside]{article}%para dar formato al documento
\usepackage{amsmath}
\usepackage[spanish]{babel}
\usepackage{graphicx}
\usepackage{xcolor}
\usepackage[utf8]{inputenc}
\usepackage{enumitem}
\title{Bitácora 13}
\author{Carlos Emiliano Sandoval Amador \\ 315025222}
\date{23 de enero de 2019}
\begin{document}
	\maketitle
	\section*{Recursividad.}
	En la clase de hoy aprendimos a definir una función de manera recursiva. Necesitamos dividir en dos casos ajenos, en el primero se da el valor que la función toma en el primer elemento y en el segundo caso el valor que toma la función en el n-ésimo valor en términos del n-1-ésimo valor. Si una función recursiva está bien definida entonces con un número finito de pasos siempre llegamos al paso base.
	\section*{Global.}
	Con la palabra reservada \textbf{global} seguido del nombre de una variable, cualquier cambio hecho a esta variable después de la palabra reservada será efectuado de manera global, es decir, sin importar si la variable está dentro de algún bloque o afuera.
	\section*{Python tutor.}
	La página \textcolor{blue}{http://pythontutor.com/visualize.htmlmode=edit} nos permite ver como cambian los valores de las variables de un código línea por línea de diversos lenguajes como python, ruby y java. Con esta página podemos identificar errores o anomalias en niestro código.
\end{document}