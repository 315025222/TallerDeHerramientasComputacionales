\documentclass[letterpaper, 12pt, oneside]{article}
\usepackage{graphicx}
\usepackage{xcolor}
\usepackage[utf8]{inputenc}
\usepackage{enumitem}
\title{Bitácora 08}
\author{Carlos Emiliano Sandoval Amador \\ 315025222}
\date{16 de enero de 2019}
\begin{document}
	\maketitle % utilizamos Large y textbf para hacer los títulos y con \\ le damos forma al texto.
	\textbf{\Large Format.} \\
	Aprendimos a ejecutar un archivo de python de versiones anteriores. También vimos una nueva forma de agregar datos a un string usando el método \textbf{.format}, para utilizarlo hay que colocar \textbf{.format} después del string seguido de un paréntesis dentro del cuál colocaremos el o los datos que queremos agregar, en lugar de usar porcentaje usamos llaves. La ventaja que tiene \textbf{.format} sobre el porcentaje es que podemos agregar datos en orden diferente, pues podemos utilizar etiquetas para cada dato y podemos colocar un dato en más de una ocasión en el string con tan solo ponerlo una vez en \textbf{format}.\\ \\ \textbf{\Large Programación orientada a objetos.} \\
	Comenzamos la parte de programación orientada a objetos, viendo los conceptos de objeto y clase.\\ \\
	 \textbf{Objeto.} \\
	  Un objeto es un conjunto de datos y funciones relacionadas que tiene atributos y métodos. Un \textbf{atributo} es una característica del objeto, por ejemplo en un auto el número de puetas o la gasolina que usa y un \textbf{método} es una acción que realiza el objeto, por ejemplo el coche acelera y frena. \\ \\
	   \textbf{Clase} \\ 
	   Una clase es una plantilla para la creación de objetos de datos según un modelo predefinido. Cada clase es un modelo que define un conjunto de variables, los atributos y métodos apropiados para operar con dichos datos. Cada objeto creado a partir de la clase se denomina instancia de la clase. \\ \\
	    \textbf{\Large Cambiar tipos de datos.} \\ 
	    Para saber el tipo de dato de una variable podemos utilizar la función \textbf{type()}. Para cambiar datos a entero utilizamos \textbf{int()}, para cambiar a string utilizamos \textbf{str()} y para cambiar a flotante utilizamos \textbf{float()}. \\ \\
	Aprendimos a hacer matrices en latex utilizando begin con el entorno matrix.
	
\end{document}