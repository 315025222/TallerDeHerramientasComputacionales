\documentclass[letterpaper, 12pt, oneside]{article}%para dar formato al documento
\usepackage{amsmath}
\usepackage[spanish]{babel}
\usepackage{graphicx}
\usepackage{xcolor}
\usepackage[utf8]{inputenc}
\usepackage{enumitem}
\title{Bitacora 05}
\author{Carlos Emiliano Sandoval Amador \\ 315025222}
\date{11 de enero de 2019}
\begin{document}
	\maketitle
	\textbf{Cadenas de texto.} \\ Vimos como utilizar cadenas de texto y agregar datos importantes a estas utilizando el caracter  \%. Para hacer un string utilizamos las comillas "" y dentro ponemos la cadena de texto que deseamos, así que para cada dato que ingresemos a la cadena de texto vamos a necesitar un caracter de \% seguido de una letra que indica el tipo de dato que queremos ingresar. Estas letras están asignadas de la siguiente manera.
	\begin{enumerate}
		\item g y f para flotantes.
		\item E y e para flotantes con notación científica.
		\item d para enteros.
		\item s para strings.
	\end{enumerate}
	Por último al final del string debemos poner \%() y dentro del paréntesis cada uno de los datos en orden separados por comillas. \\ \\ \textbf{Definición de funciones.} \\ Para definir una función tenemos que utilizar la palabra reservada def seguida del nombre de la función y unos paréntesis dentro de los cuáles están los parámetros de la función.
	
 	

\end{document}