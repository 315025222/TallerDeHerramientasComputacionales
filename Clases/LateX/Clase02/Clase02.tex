\documentclass[letterpaper, 12pt, oneside]{article}%para dar formato al documento
\usepackage{amsmath}
\usepackage[spanish]{babel}
\usepackage{graphicx}
\usepackage{xcolor}
\usepackage[utf8]{inputenc}
\usepackage{enumitem}
\title{Bitacora 02}
\author{Carlos Emiliano Sandoval Amador \\ 315025222}
\date{08 de enero de 2019}



\begin{document}
	\maketitle
	\section{Comandos.}
	{En la segunda clase del curso aprendimos los comandos de bash, el interprete de comandos de linux, para poder movernos, crear directorios, visualizar el contenido de estos y ejecutar porgramas con el editor de textos vi.\\ \\ A continuación daremos una breve explicación de los comandos más importantes.\\}
	\subsection{ls.}
	{El comando ls sirve para mostrar los directorios y archivos contenidos en el directorio que le indiques. Si no le indicas un directorio, te mostrará el contenido del directorio en el que estés parado. \\ Utilizando ls -a verás una lista de todos los archivos incluidos los archivos ocultos.}
	\subsection{cd.}
	{Con el comando cd puedes moverte a directorios que estén contenidos en el directorio en el que te encuentras. Si no especificas un directorio cd te regresará al directorio home, este contiene a todos los demás directorios. Con cd .. puedes moverte al directorio que contiene al directorio en el que te encuentras.}
	\subsection{man.}
	{Muestra el manual del comando que desees, de esta forma puedes aprender a utilizarlos o descubrir funciones que antes no conocias.}
	\subsection{mkdir.}
	{Crea un nuevo directorio contenido en el directorio en el que te encuentres. Utilizando la extensión -p puedes crear directorios anidados.}
	\subsection{rmdir.}
	{Elimina un directorio contenido en el directorio en el que te encuentres siempre y cuando este esté vacío.}
	\subsection{history.}
	{Muestra una lista con los comandos usados en la última sesión.}
	\subsection{touch.}
	{Con el comando touch puedes crear un archivo contenido en el directorio en el que estés. Solo tienes que escribir touch seguido del nombre del archivo que deseas crear.}
	\subsection{less.}
	{El comando less muestra el contenido de un archivo. Si el archivo es binario el contenido del archivo mostrado por less será imposible de entender por el ser humano.}
	
	
	
\end{document}